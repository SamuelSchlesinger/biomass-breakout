\documentclass{article}
\usepackage{amsmath}
\usepackage{amssymb}

\title{Biomass Breakout: An Evolutionary Simulation Game}
\author{Samuel Schlesinger}
\date{\today}

\begin{document}

\maketitle

\section*{Concept}

Biomass Breakdown is all about giving players a way to experiment with
evolution. Players begin the game by creating a population of creatures which
evolve over time based on the pressures in their environments, which the
players will have partial control over. There will be two modes, experimental
and competitive, but this document currently will only cover experimental
as competitive will be influenced a lot from our experimentation using and
building the experimental mode.

The idea is that a tank of biomass has been cracked open on a planet which
formerly failed as a terraforming project. Once a world meant to give humanity
a new life, this world will become a new life from whatever species develop
from the oozing tank. Here, we really want to emphasize the word
\textbf{develop} rather than \textbf{create}, as we hope to use the word
evolution quite literally.

With this goal in mind, the player's direct control of the population will be
completely forbidden. Instead, the player will be able to control the
environment where the population exists, growing it from a small, nutrient rich
plot to a sprawling ecological paradise with irregular nutrient growth. As we
believe that cooperative behavior will more likely be seen in surplus, we would like
to use this to our advantage in creating an early game consisting of primarily
non-predatory behaviors, and introduce scarcity, which we believe to be an incentive
for predatory behaviors, slowly as the game proceeds. We believe that this
will be more likely to create creatures which construct complex ecological systems
early game.

\end{document}
